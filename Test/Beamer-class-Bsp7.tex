% Dieser Text ist urheberrechtlich gesch�tzt
% Er stellt einen Auszug eines von mir erstellten Referates dar
% und darf nicht gewerblich genutzt werden
% die private bzw. Studiums bezogen Nutzung ist frei
% Mai. 2007
% Autor: Sascha Frank 
% Universit�t Freiburg 
% www.informatik.uni-freiburg.de/~frank/
\documentclass{beamer}
\setbeamertemplate{navigation symbols}{}
\usepackage{beamerthemeshadow}
\beamersetuncovermixins{\opaqueness<1>{25}}{\opaqueness<2->{15}}
\begin{document}
\title{Beamer Class ganz nett}  
\author{Sascha Frank}
\date{\today} 

\begin{frame}
\titlepage
\end{frame} 

\begin{frame}
\frametitle{Inhaltsverzeichnis}\tableofcontents
\end{frame} 


\section{Abschnitt Nr.1} 
\begin{frame}
\frametitle{Titel} 
Die einzelnen Frames sollte einen Titel haben 
\end{frame}
\subsection{Unterabschnitt Nr.1.1  }
\begin{frame} 
Denn ohne Titel fehlt ihnen was
\end{frame}


\section{Abschnitt Nr. 2} 
\subsection{Listen I}
\begin{frame}\frametitle{Aufz\"ahlung}
\begin{itemize}
\item Einf\"uhrungskurs in \LaTeX  
\item Kurs 2  
\item Seminararbeiten und Pr\"asentationen mit \LaTeX 
\item Die Beamerclass 
\end{itemize} 
\end{frame}

\begin{frame}\frametitle{Aufz\"ahlung mit Pausen}
\begin{itemize}
\item  Einf\"uhrungskurs in \LaTeX \pause 
\item  Kurs 2 \pause 
\item  Seminararbeiten und Pr\"asentationen mit \LaTeX \pause 
\item  Die Beamerclass
\end{itemize} 
\end{frame}

\subsection{Listen II}
\begin{frame}\frametitle{Numerierte Liste}
\begin{enumerate}
\item  Einf\"uhrungskurs in \LaTeX 
\item  Kurs 2
\item  Seminararbeiten und Pr\"asentationen mit \LaTeX 
\item  Die Beamerclass
\end{enumerate}
\end{frame}
\begin{frame}\frametitle{Numerierte Liste mit Pausen}
\begin{enumerate}
\item  Einf\"uhrungskurs in \LaTeX \pause 
\item  Kurs 2 \pause 
\item  Seminararbeiten und Pr\"asentationen mit \LaTeX \pause 
\item  Die Beamerclass
\end{enumerate}
\end{frame}

\section{Abschnitt Nr.3} 
\subsection{Tabellen}
\begin{frame}\frametitle{Tabellen}
\begin{tabular}{|c|c|c|}
\hline
\textbf{Zeitpunkt} & \textbf{Kursleiter} & \textbf{Titel} \\
\hline
WS 04/05 & Sascha Frank &  Erste Schritte mit \LaTeX  \\
\hline
SS 05 & Sascha Frank & \LaTeX \ Kursreihe \\
\hline
\end{tabular}
\end{frame}


\begin{frame}\frametitle{Tabellen mit Pause}
\begin{tabular}{c c c}
A & B & C \\ 
\pause 
1 & 2 & 3 \\  
\pause 
A & B & C \\ 
\end{tabular} 
\end{frame}


\section{Abschnitt Nr. 4}
\subsection{Bl\"ocke}
\begin{frame}\frametitle{Bl\"ocke}

\begin{block}{Blocktitel}
Blocktext 
\end{block}

\begin{exampleblock}{Blocktitel}
Blocktext 
\end{exampleblock}


\begin{alertblock}{Blocktitel}
Blocktext 
\end{alertblock}
\end{frame}

\section{Abschnitt Nr. 5}
\subsection{Geteilter Bildschirm}

\begin{frame}\frametitle{Zerteilen des Bildschirmes}
\begin{columns}
\begin{column}{5cm}
\begin{itemize}
\item Beamer 
\item Beamer Class 
\item Beamer Class Latex 
\end{itemize}
\end{column}
\begin{column}{5cm}
\begin{tabular}{|c|c|}
\hline
\textbf{Kursleiter} & \textbf{Titel} \\
\hline
Sascha Frank &  \LaTeX \ Kurs 1 \\
\hline
Sascha Frank & \LaTeX \ Kursreihe \\
\hline
\end{tabular}
\end{column}
\end{columns}
\end{frame}

\subsection{Bilder} 
\begin{frame}\frametitle{Bilder in Beamer}
\begin{figure}
\includegraphics[scale=0.5]{PIC1} 
\caption{Die Abbildung zeigt ein Beispielbild}
\end{figure}
\end{frame}


\subsection{Bilder und Listen kombiniert} 

\begin{frame}
\frametitle{Bilder und Itemization in Beamer}
\begin{columns}
\begin{column}{5cm}
\begin{itemize}
\item<1-> Stichwort 1
\item<3-> Stichwort 2
\item<5-> Stichwort 3
\end{itemize}
\vspace{3cm} 
\end{column}
\begin{column}{5cm}
\begin{overprint}
\includegraphics<2>{PIC1}
\includegraphics<4>{PIC2}
\includegraphics<6>{PIC3}
\end{overprint}
\end{column}
\end{columns}
\end{frame}

\subsection{Bilder die mehr Platz brauchen} 
\begin{frame}[plain]
\frametitle{plain, oder wie man mehr Platz hat}
\begin{figure}
\includegraphics[scale=0.5]{PIC1} 
\caption{Die Abbildung zeigt ein Beispielbild}
\end{figure}
\end{frame}







\end{document}