\documentclass{article}

\author{Hugh C. Pumphrey}

\title{Alternative fonts in \LaTeX: a test document}

%% This one is OK apart from Fn names.
%%\usepackage[LGR,T1]{fontenc} %% LGR encoding is needed for loading the package gfsneohellenic
%%\usepackage[default]{gfsneohellenic}

%% Works well. A bit whacky. Not in all installations.
%%\usepackage[T1]{fontenc}
%%\usepackage{gfsartemisia-euler}

%% Good sans font for in presentations. Not in all installations.
%\usepackage[T1]{fontenc}
%\usepackage{arev}

%  URW Nimbus Roman (A Times clone).
%\usepackage[T1]{fontenc}
%\usepackage{mathptmx}

%% Helvetica (actually URW Nimbus Sans (A Helvetica clone)). 
\usepackage[T1]{fontenc}
\usepackage[scaled]{helvet}
\renewcommand*\familydefault{\sfdefault} %% Only if the base font of
%the document is to be sans serif
%% This is a glaring example of not having math support, unless you do this:
%% Note the use of EULERGREEK to ensure that you get some greek
%% letters that do not look too out of place
\usepackage[EULERGREEK]{sansmath}
\sansmath


%TeX Gyre heros ( \usepackage{tgheros} ) is a better helvetica
%package, (e.g. \Delta works) but UofE computers don't have it.

%% Ariel. No math support here either. NERC insist on this for
%% proposals. Really URW A030 (An Arial clone).
%% Not available on stock Debian. No longer on School machines either,
%% probably as it is non-Free

%\usepackage[T1]{fontenc}
%\usepackage[scaled]{uarial}
%\renewcommand*\familydefault{\sfdefault} %% Only if the base font of the document is to be sans serif


\begin{document}
\maketitle

\section{Introduction}

The vast majority of \LaTeX\ documents use the standard computer
modern fonts. But you do not have to stick with these old stalwarts:
there are some alternative font possibilities. A word of caution,
though. If you want your document to look nice, all the fonts in it
should work together and look good together. Or you will get lossage
such as the numbers in formul\ae\ not looking like those in the
running text. This document contains many of the style commands common
in \LaTeX\ documents to test whether a given font package supports
them all. For more information, go to the \LaTeX\ font catalogue at
\texttt{http://www.tug.dk/FontCatalogue/}.

\section{Some tests}
\subsection{Font styles}

It is common in a \LaTeX\ document to use 
\begin{itemize}
\item \verb+\emph{}+ to \emph{emphasise} text, 
\item \verb+\textit{}+ to make \textit{italic} text,
\item \verb+\textsl{}+ to make \textsl{slanted} text,
\item \verb+\textbf{}+ to make \textbf{bold} text,
\item \verb+\textsf{}+ to get \textsf{sans serif}, 
\item \verb+\texttt+ for a \texttt{monospaced typewriter font.} 
\item \verb+\textsc{}+ \textsc{To Speak Like
  Death, in CAPS and small caps.} 
\item  \verb+\textrm{}+ \textrm{to get text in a serif font?}
\end{itemize}
Note that \textit{italic} and \textsl{slanted} are the same as each
other in some font packages and different from each other in other
font packages. Also note that if your main font is a sans font then
\verb+\textsf{}+ and  \verb+\textrm{}+ are the same thing.



\subsection{Mathematical formul\ae}

In-line equations look like this: $x^2 + y^2 = z^2$ Displayed
equations look like this:
%
\[ A = \int_0^\infty \frac{x^2 \cos ax}{1+x^3}\, dx \]
%
Numbered equations are done using \verb+\begin{equation}+ and
  \verb+\end{equation}+, like in equation~\ref{eq:example}.
\begin{equation}
\label{eq:example}
\sigma = \sqrt{  \frac{1}{n}\sum_{j=0}^n (x_j - \bar x)^2 }
\end{equation}
%
To ensure that lower-case greek letters such as $\lambda$, $\alpha$, $\beta$,
$\theta$, $\phi$, and upper-case greek letters such as $\Delta$,
$\Theta$, $\Phi$, $\Xi$ appear correctly we
try this: 
\[ \cos \theta \sin\phi = \cos\Delta_C \sin\Delta_A - \tan \Psi \cot
\Xi \]
There are some font packages in which the capital Greek letters do not
appear correctly.
%
%
A problem with alternative fonts is that 1234567890 (running text) may
look different from $1234567890$ (in math mode). They should look the
same. Here they are next to each other, so that you can
check:\\ 1234567890$1234567890$\\ Another problem is that the function
names in formul\ae\ may revert to computer modern
roman. Compare\\ $\cos x + \sin y$ in math mode with\\ cos \emph{x} +
sin \emph{y} in text mode. Text mode won't get the spacing right (and
maybe not the fonts for $x$ and $y$), but the fonts for cos and sin
should match.



\end{document}






