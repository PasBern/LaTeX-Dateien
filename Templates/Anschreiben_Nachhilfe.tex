\documentclass[11pt,german]{g-brief}
\usepackage[latin1]{inputenc}
\usepackage{pdfpages}
\usepackage{mathpazo}
 
 
\fenstermarken \trennlinien
 
\Name {Pascal Bernhard}
\Strasse {Schwalbacherstra\ss{}e 7}
\Ort {12161 Berlin}
\Telefon {0152-38502363}
%\Telex{Faxnummer}
%\HTTP {http://www.maxis-welt.de}
\EMail{pascal.bernhard@belug.de}
\Unterschrift {Pascal Bernhard}
 
\Adresse { Studienkreis\\- Personalabteilung -\\
		     Landstra\ss{}e 2\\12345 Musterstadt}
 
\Betreff {Bewerbung als ...} \Datum {\today}
\Anrede {Sehr geehrte Damen und Herren,}
\Gruss {Mit freundlichen Gr\"{u}\ss{}en}{1cm}
 
\begin{document}
\begin{g-brief}
hier folgt der erste Absatz, der auch gleichzeitig die
\textbf{Einleitung} darstellt. Am besten kommt man gleich zur
Sache: Warum interessiert mich diese Stelle, und warum halte
ich mich f"ur geeignet.
 
Im zweiten Absatz beginnt der \textbf{ Hauptteil}. Hier stellt
man sich vor, und hier sollte man anhand von Qualifikationen
und Erfahrungen belegen, warum man die Anforderungen
erf"ullt. Im Hauptteil sollte man auch pers"onliche Qualit"aten
erw"ahnen: Welche Hard und Soft Skills bringe ich mit (ich bin
teamf"ahig, flexibel, etc.).
 
Der letzte Absatz geh"ort dem \textbf{Schluss}. Hier bekundet
man nocheinmal sein Interesse, sowie die Reaktion, die man sich
w"unscht. ("`"Uber eine Einladung zu einem pers"onlichen
Gespr"ach w"urde ich mich sehr freuen."')
\end{g-brief}
\end{document}