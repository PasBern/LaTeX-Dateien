% Created 2012-01-14 Sat 01:59
\documentclass[11pt,a4paper,ngerman]{article}
\renewcommand{\familydefault}{\sfdefault}
\usepackage[T1]{fontenc}
\usepackage[utf8]{inputenc}
\usepackage{fancyhdr}
\pagestyle{fancy}
\usepackage[ngerman]{babel}
\usepackage[unicode=true,pdfusetitle,
 bookmarks=true,bookmarksnumbered=false,bookmarksopen=false,
 breaklinks=true,pdfborder={0 0 
0},backref=section,colorlinks=true,urlcolor=blue]{hyperref}
\usepackage{breakurl}
\usepackage{lastpage}
\usepackage{textcomp}
\usepackage[official]{eurosym}

%%%keine Monospace-Schriftart für Hyperlinks
\newcommand{\urlwofont}[1]
{
\urlstyle{same}\url{#1}
}


%%%verkleinert den Abstand bei Listen und Aufzählungen zwischen denm einzelnen 
%%%Punkten
\usepackage{mdwlist}
\usepackage{xcolor}


%%Definition von Grautoenen
\definecolor{dunkelgrau.80}{gray}{0.20}
\definecolor{hellgrau.60}{gray}{0.40}


% Kopf- und Fusszeile
\renewcommand{\headrulewidth}{0.4pt}
\fancyhf{}

\lhead{Protokoll - Mitgliederversammlung 12.Juni 2013}
\rhead{\thepage \ | \pageref{LastPage}}
\cfoot{}


\title{\textbf{Mitgliederversammlung - Protokoll}}
\author{\emph{Berlin Linux User Group}}
\date{12. Juni 2013}


%%%%%%%%%%%%%%%%%%%%%%%%%%%%%%%%%%%%%%%%%%%%%%%%%%%%%%%%%%%%%%%%%%%%%%%%%%%%%%%%


\begin{document}
\selectlanguage{ngerman}

\maketitle
\thispagestyle{empty}
\newpage


\setcounter{tocdepth}{2}
\tableofcontents

\newpage




\section{TOPIC I: Anwesenheit/Beschlussfähigkeit/Wahl des Versammlungsleiters 
und Protokollführers}

  \subsection{Anwesende Mitglieder}

    \begin{itemize*}
      \item Lutz Matscholl
      \item Bodo Eichstädt
      \item Klaus Montigel
      \item Norbert Ziese
      \item Andreas Gläser
      \item Pascal Bernhard
      \item Reinhard Peiler
      \item Friedrich W. Brockstedt
      \item Rainer Herrendörfer
      \item Frank Hildebrecht
      \item Sebastian Andres
      \item Ralf Vögtle
      \item Lutz Willek
      \item Gerhard Lüdtke
      \item Rüdiger Harnisch
      \item Christoph Koydl
      \item Claus Schäfer
      \item Arne Linus Eichstädt
      \item Philipp von den Linden
      \item Michael Rößler
      \item Thorsten Stöcker


   \end{itemize*}

    Anzahl: 20

  \subsection{Beschlussfähigkeit \& Wahl von Versammlungsleiter und 
	      Protokollführer}

	      Der Vorstandsvorsitzende stellte fest, dass die 
	      Mitgliederversammlung satzungsgemäß einberufen wurde und 
	      beschlussfähig ist. \\
	      \\
	      Zum Versammlungsleiter wurde \emph{Friedrich W. Brockstedt} 
	      bestimmt. \\
	      Zum Protokollführer wurde \emph{Pascal Bernhard} bestimmt.



  \subsection{Genehmigung der Tagesordnung}

	      Es wurden keine Themen zur Tagesordnung hinzugefügt. \\
	      Die Tagesordnung wurde einstimmig angenommen.

\newpage{}	  
%%%%%%%%%%%%%%%%%%%%%%%%%%%%%%%%%%%%%%%%%%%%%%%%%%%%%%%%%%%%%%%%%%%%%%%%%%%%%%%%
	
	
\section{TOPIC II: Berichte zu Aktivitäten im 1. Halbjahr 2013}

  
  \subsection{Document Freedom Day [von: 
              \textcolor{hellgrau.60}{\textsl{Andreas Gläser, Lutz Matscholl}}]}

              
Den \textsl{Document Freedom Day 2013} können wir als insgesamt 
gelungen betrachten. Neben einem Vortrag von Andreas Gläser zum Thema 'Freie 
Dokumentformate' kam bei den Besuchern auch das gemeinsame Kochen und Essen gut 
an. Leider hatten wir keine Unterstützung durch die \emph{Free Software 
Foundation Europe} (FSFE), was möglichweise daran lag, dass wir sie zu spät 
kontaktiert hatten.



  \subsection{LinuxTag 2013 [von: \textcolor{hellgrau.60}{\textsl{Ralf 
              Vögtle}}]}
	    
Dank sehr guter Personalorganisation und engagierten Mitgliedern können wir 
unsererseits mit der Unterstützung der Organisatoren des LinuxTags zufrieden 
sein. Kein einziger Helfer ist während der vier Messetage ausgefallen, auch war 
unser Stand stets mit einer angebrachten Anzahl von Mitgliedern besetzt. Der 
RepRap von Uwe (IN-Berlin) zog viele Messebesucher an unseren Stand und auch 
Haukes GPG-Programm traf auf Interesse. Leider haben wir auch dieses Jahr 
wieder kein eigenes, überzeugendes Konzept der Eigendarstellung auf 
öffentlichen Veranstaltungen aufstellen können. Einerseits war der Stand für 
Messebesucher wenig attraktiv, sprich die Präsentation der BeLUG als Verein war 
nicht zufriedenstellend. Zudem machte unser Stand einen unaufgeräumten Eindruck 
(Beispielsweise standen die leeren Bierkisten nach der Standparty am ersten 
Messertag die restlichen Tage weiterhin sichtbar am Stand herum). Die Anzahl 
der Brezeln bei der Standparty war nicht ausreichend.
  
  
  
  \subsection{Mitgliedsbeiträge [von: \textcolor{hellgrau.60}{\textsl{Reinhard 
              Peiler}}]}

Leider sind ca. 30 Mitglieder im Rückstand mit ihren Mitgliedsbeiträgen. Diese 
Mitglieder werden vom Vorstand informiert und entrichten hoffentlich ihren 
ausstehende Mitgliedsbeitrag. Sollte dies nicht geschehen, verlieren die 
Betroffenen ihren Mitgliedstatus nach §4 der Satzung:\par

\vspace{1cm}

\textcolor{dunkelgrau.80}{\textit{\textbf{\S4 Ausscheiden von Mitgliedern}
[...] Die Mitgliedschaft erlischt mit dem Austritt eines Mitglieds, der nur 
jeweils zum Quartalsende möglich ist und unter Einhaltung einer Frist von einem 
Monat zum Quartalsende schriftlich erklärt werden muss. Ein Mitglied kann von 
der Mitgliederliste gestrichen werden, wenn es trotz zweimaliger Mahnung mit 
der Zahlung des Beitrags im Rückstand ist. Die Streichung darf erst beschlossen 
werden, nachdem seit der Absendung des zweiten Mahnschreibens drei Monate 
verstrichen und die Beitragsschulden nicht beglichen sind. Die Streichung ist 
dem Mitglied mitzuteilen.Ein ausgetretenes Mitglied hat keinen Anspruch gegen 
das Vereinsvermögen.}}\\
\\
\noindent
Auf der nächsten Mitgliederversammlung wird der Vorstand das Resultat der 
Aufarbeitung der Mitgliederliste bekannt geben.


  \subsection{Bericht des Kassenwarts [von: 
\textcolor{hellgrau.60}{\textsl{Frank Hildebrecht}}]}

Gegenwärtig befinden sich auf dem Bankkonto der BeLUG 3842,98 EUR, in der Kasse 
189,05 EUR. In den nächsten Monaten wird das Konto von der Postbank zur 
Gemeinschaftsbank für Leihen und Schenken (kurz: GLS Bank) umgezogen. Bei den 
Zahlungsmodalitäten des Mitgliedsbeitrages wird es Änderungen zur 
bisherigen Praxis geben (Diese wurden per Mehrheitsbescheid von den Mitgliedern 
beschlossen). Die bargeldlose Entrichtung des Mitgliedsbeitrages wird weiterhin 
möglich bleiben. Der Beitrag muss, gleich ob per Überweisung oder mit Bargeld 
bis Ende Februar des betreffenden Jahres gezahlt werden. Neue Mitglieder müssen 
innerhalb von 4 Wochen der erfolgreichen Antragstellung auf Aufnahme in 
den Verein ihren anteiligen Beitrag entrichten.


  \subsection{Bericht zum Server [von: 
\textcolor{hellgrau.60}{\textsl{Lutz Willek}}]}

Dank einer zweckgebundenen Spende von \emph{Six Minutes Media} und einer 
weiteren Spende der \emph{qipu GmbH} von gesamt über 1700 Euro konnte endlich 
für die BeLUG ein neuer Server angeschafft werden. Dieser neue Server löst den 
Desktop-Rechner ab, der bisher für diesen Zweck verwendet wurde. Neben den 
verbesserten Verwaltungsmöglichkeiten ist dieser Server deutlich 
leistungsfähiger und gleichzeitig stromsparender als sein Vorgänger. Der Server 
kostete etwa 800 Euro, spart im Vergleich zum vorherigen Setup etwa 50 W ein. 
Der Desktop-Rechner konnte einem anderen Zweck zugeführt werden. Herzlichen 
Dank an die Spender, die mit Ihrer Spende den neuen Server erst ermöglicht 
haben.


  \subsection{Bericht zur Küche [von: 
\textcolor{hellgrau.60}{\textsl{Lutz Matscholl}}]}

Leider mussten wir feststellen, dass der Abfluss in der Küche zum unzähligsten 
Male verstopft ist. Gerhard Lüdtke ist nach beträchtlichem Zeit- \& Geldeinsatz 
nicht mehr bereit, die Pumpe ein weiteres Mal zu reparieren. Jetziger Plan: 
zusammen mit dem IN-Berlin soll eine Tür oder ähnliches installiert werden, 
damit die Küche abgesperrt werden kann. Der Schlüssel hierfür soll nur 
kontrolliert herausgegeben werden.


  \subsection{Bericht zu Buchrezensionen [von: 
\textcolor{hellgrau.60}{\textsl{Claus Schäfer}}]}

Die Buchrezensionen sind für die BeLUG sehr einträglich, da sie für viele 
Besucher der Einstieg zu unserer Webseite sind, das heißt sie kommen auf der 
Suche nach einer speziellen Buchbesprechung zu uns. Leider stehen weiterhin 
drei Buchrezensionen aus, die noch abgeliefert werden müssen.


  \subsection{Bericht zu kommenden Vorträgen [von: 
\textcolor{hellgrau.60}{\textsl{Sebastian Andres}}]}

Für das zweite Halbjahr 2013 sind zwei Vorträge fest eingeplant mit den Themen: 
\textsl{OpenWRT} und \textsl{Solaris}.

%%%%%%%%%%%%%%%%%%%%%%%%%%%%%%%%%%%%%%%%%%%%%%%%%%%%%%%%%%%%%%%%%%%%%%%%%%%%%%%%

\section{TOPIC III: Geplante Aktivitäten im 2. Halbjahr 2013}


  \subsection{Helfergrillen}
  
Am 19. Juni wird das Helfergrillen in der Kulturfabrik stattfinden, um uns so 
bei den Helfern am LinuxTag für ihren Einsatz zu bedanken. Thorsten Stöcker 
bietet seine Hilfe bei der Vorbereitung des Helfergrillens an, obwohl er 
selbst nicht am LinuxTag präsent war.



  \subsection{Software Freedom Day}
  
Unser Engagement am Software Freedom Day (21. September) hängt von der 
Unterstützung der \emph{Free Software Foundation Europe} (FSFE) ab, das heißt 
ob von ihrer Seite ein Workshop angeboten wird oder nicht.


  \subsection{Linux Install Partys an Berliner Universitäten}

Dieses Jahr möchten wir neben der Unterstützung für Linux Install Partys an der 
Humboldt-Universität und der Technischen Universität auch der Linux-Gruppe 
SPLINE (Studentisches Projekt Linux Netzwerke) der Freien Universität unsere 
Hilfe anbieten.

%%%%%%%%%%%%%%%%%%%%%%%%%%%%%%%%%%%%%%%%%%%%%%%%%%%%%%%%%%%%%%%%%%%%%%%%%%%%%%%%

\section{TOPIC IV: Projekte}


  \subsection{Projekt: Raspberry Pi als Multimedia-Zentrale}
  
Für dieses Projekt ist die Anschaffung eines \emph{Raspebeery Pis} samt 
Speicherkarte und Gehäuse für insgesamt ca. 60 EUR geplant. Das Ergebnis soll 
auf dem großen Fernseher gezeigt werden. Ein Workshop ist angedacht.


  \subsection{Projekt: eLAB}
  
Das Projekt \emph{eLAB} befindet sich derzeit noch im Aufbau, bisher treffen 
sich die Interessenten dienstags und freitags. Handicap ist die fehlende 
Institutionalisierung des Projekts, das \emph{eLAB} ist kein Verein.


  \subsection{Projekt: Spacenet}

\urlwofont{https://www.in-berlin.de/provider/spacenet.html}\\
\\
Ein Radiusserver soll eingerichtet werden, der allen BeLUG-Mitgliedern 
weltweit einfachen Internetzugang in Hackerspaces bereitstellt. Gleichzeitig 
soll ein WLAN-Zugangspunkt in den BeLUG-Räumen aufgebaut werden, der für 
Besucher mit eigener Spacenet-Kennung einen Internet-Zugang bietet. Dank der 
von Lutz Willek gespendeten Hardware sind die Voraussetzungen für die Arbeit an 
diesem Projekt gegeben. Jedoch haben weder Lutz Willek als Initiator, noch 
Sebastian Andres und Ralf Vögtle aufgrund ihrer Übernahme des Projektes 
\emph{Printserver} Zeit und suchen Mitglieder, welches dieses für die BeLUG sehr 
wichtige Projekt starten können. Sowohl der IN-Berlin, als auch Lutz Willek 
haben Hilfe bei eventuell auftauchenden Problemen zugesagt. An diesem Projekt 
interessierte Mitglieder melden sich bitte beim Vorstand, der dann die 
Bereitstellung der Hardware organisiert und die Zugangsdaten zu \emph{Spacenet} 
weitergibt.



  \subsection{Projekt: Printserver}

Die Betreuung des Projektes werden Ralf Vögtle, Philipp von der Linden und 
Sebastian Andres übernehmen. Ziel ist die Weiterentwicklung des Webinterfaces, 
da Jean-Christoph Duberga diese aus Zeitmangel nicht mehr übernehmen kann.  
  
  
  
  \subsection{Webseite der BeLUG}
  
Das Design der Vereinswebseite wird durch Fred Brockstedt und Pascal Bernhard 
neu gestaltet. Ebenso muss die inhaltliche Struktur der Internetseite 
überarbeitet werden, was Reinhard Peiler und Pascal Bernhard in Angriff nehmen 
werden. Dauerauftrag und Aufgabe für alle Mitglieder der BeLUG ist die 
inhaltliche Pflege der Webseite, deren Content regelmäßig aktualisiert werden 
muss. Hierfür werden noch Mitglieder zur Unterstützung gesucht. 
  
  

  \subsection{Themenabende}

Fred Brockstedt bietet technische Unterstützung für die Veranstaltung und 
Leitung von Themenabenden an. Generell sind alle Mitglieder aufgerufen, in 
Erwägung zu ziehen, einen Themenabend zu einem Thema ihrer Wahl zu leiten.

%%%%%%%%%%%%%%%%%%%%%%%%%%%%%%%%%%%%%%%%%%%%%%%%%%%%%%%%%%%%%%%%%%%%%%%%%%%%%%%%


\section{TOPIC V: Verschiedenes}


In Zukunft sollen die Vereinsräume regelmäßig und gemeinsam von den 
Mitgliedern aufgeräumt und gepflegt werden.\\
\\
\\
\textbf{Die Mitgliederversammlung endet um 20:58h.}








\end{document}