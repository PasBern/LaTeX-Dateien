\documentclass[11pt,a4paper,titlepage]{moderncv}

% moderncv themes
\moderncvtheme[blue]{casual}                 % optional argument are 'blue' (default), 'orange', 'red', 'green', 'grey' and 'roman' (for roman fonts, instead of sans serif fonts)
%\moderncvtheme[green]{classic}                % idem

% character encoding
\usepackage[english,ngerman]{babel}
\usepackage[ansinew]{inputenc}                   % replace by the encoding you are using

% adjust the page margins
\usepackage[scale=0.8]{geometry}
\usepackage{graphicx}
%\setlength{\hintscolumnwidth}{3cm}						% if you want to change the width of the column with the dates
%\AtBeginDocument{\setlength{\maketitlenamewidth}{6cm}}  % only for the classic theme, if you want to change the width of your name placeholder (to leave more space for your address details
\AtBeginDocument{\recomputelengths}                     % required when changes are made to page layout lengths



%----------------------------------------------------------------------------------
%            Kontaktdaten
%----------------------------------------------------------------------------------
% VORNAME
\firstname{Pascal}
% NACHNAME
\familyname{Bernhard}
%TITEL (optional, ggf. einfach die Zeile l�schen!)
\title{Lebenslauf}    
%ADRESSE  (optional, ggf. einfach die Zeile l�schen!)
\address{Schwalbacher Strasse 7}{12161 Berlin}
%HANDYNUMMER  (optional, ggf. einfach die Zeile l�schen!)
\mobile{0177 / 88 10 80 2} 
%FESTNETZNUMMER  (optional, ggf. einfach die Zeile l�schen!)
\phone{030 / 32 66 58 00}
%EMAIL-ADRESSE  (optional, ggf. einfach die Zeile l�schen!)
\email{pascal.bernhard@belug.de} 
%HOMEPAGE  (optional, ggf. einfach die Zeile l�schen!)
\homepage{www.secunergy.com}
%FOTO  (optional, ggf. einfach die Zeile l�schen!)
%  64pt = H�he des Bildes, 'picture' = Name des Bildes
\photo[120pt]{picture}

% to show numerical labels in the bibliography; only useful if you make citations in your resume
%\makeatletter
%\renewcommand*{\bibliographyitemlabel}{\@biblabel{\arabic{enumiv}}}
%\makeatother

%Spaltenbreite von \cvline �ndern:
\renewcommand*{\cvlanguage}[3]{%
\cvline{#1}{\begin{minipage}[t]{.225\maincolumnwidth}\textbf{#2}\end{minipage}%
\hfill%
\begin{minipage}[t]{.725\maincolumnwidth}\raggedleft\footnotesize\it shape #3\end{minipage}%
}
}

%Spaltenbreite von \cvline �ndern - zweiter Versuch
\setlength{\hintscolumnwidth}{3.5cm} 


%----------------------------------------------------------------------------------
%            Inhalt
%----------------------------------------------------------------------------------
\begin{document}


    \vspace*{\fill}
    
      \maketitle
      \section{Zu meiner Person}
	\cvline{Geboren}{am 08.04.1982 in Berlin}
	\cvline{Studienfach}{Politikwissenschaft}
	\cvline{Auslandserfahrung} {(USA \& Frankreich)}
	\cvline{Sprachkenntnisse}{Englisch \& Franz�sisch (verhandlungssicher)}
	\cvline{IT}{Linux, Office-Suite (MS Office, OpenOffice), CMS, LaTeX}

    \vspace*{\fill}



\newpage



\section{Praktische Erfahrungen}
\cvline{6/2009 - 9/2011}{Werkstudent \newline bei SCI Verkehr GmbH  in Berlin}
\cvline{10/2008 - 4/2009}{Praktikant \newline am Jean-Monnet Centre for European Integration in Berlin}
\cvline{7/2006 - 6/2007}{Praktikant \newline bei der American Chamber of Commerce in Paris}
\cvline{4/2005 - 10/2005}{Praktikant \newline am Wissenschaftszentrum f�r Sozialforschung Berlin}



\section{Studium}
\subsection{Hauptstudium "`Politikwissenschaft"'}
\cvline{9/2009 - 8/2011}{an der FU Berlin}



\subsection{Hauptstudium "`Master Finance et Strat�gie"'}
\cvline{9/2005 - 8/2007}{an Science Po Paris}



\subsection{Grundstudium "`Politikwissenschaft"'}
\cvline{9/2002 - 8/2005}{an der FU Berlin \emph{Note: 1,6}}
\cvline{13.06.2009}{Vordiplom}
\cvline{Titel}{\emph{Transformationsprozesse in Lettland}}
\cvline{Betreuer}{Dr. Prof. Kerner}


\section{Sprachkenntnisse}
\cvline{Deutsch}{Muttersprache}
\cvline{Englisch}{Verhandlungssicher in Wort und Schrift}
\cvline{Franz�sisch}{Verhandlungssicher in Wort und Schrift}
\cvline{Russisch}{Grundkenntnisse}


\section{Schulbildung}
\cvline{8/1992 - 7/2001}{Gymnasium Freiburg im Breisgau}
\cvline{8/1988 - 7/1992}{Grundschule Freiburg im Breisgau}

%\section{Sonstige Erfahrungen}
%\subsection{Praktika}
%----------------------------------------------------------------------------------
%            ACHTUNG: Bei �nderungen das "}" am Ende nicht vergessen!!!
%----------------------------------------------------------------------------------
%\cvline{2/2011 - 7/2011}{Web Developer \newline bei YourMum GmbH in Mannstedt \newline
%												 T�tigkeiten:
%												 % Aufz�hlung
%												 \begin{itemize}
%													\item %Entwicklung von Prototypen auf der Basis von PHP, (X)HTML, JavaScript (jQuery) und MySQL
%													\item %Entwicklung von Applikationen mittels PHP und XML
% 											 \end{itemize}
%}%-- Geh�rt zu \cvline
%\cvline{2/2011 - 7/2001}{Typo3 Developer \newline bei BummbBumm KG in Mannstedt \newline
%												 T�tigkeiten:
%												 % Aufz�hlung
%												 \begin{itemize}
%													\item %Optimierung der Templates und Struktur
%													\item %Beratung und Unterst�tzung der Redaktion, Vermarktung und der Marketingabteilung
%  											 \end{itemize}
%}%-- Geh�rt zu \cvline

%\subsection{Sonstiges}
%\cvline{11/2006 - 12/2008}{IT-Helpdesk Servicekraft \newline bei Musterverwaltung in Musterstadt}


\section{IT-Kenntnisse}
\cvline{}{GNU/Linux}
\cvline{}{MS Office Suite}
\cvline{}{LaTeX}
\cvline{}{Content Management Systeme (Joomla \& Contao)}


\section{Interessen}
\renewcommand{\listitemsymbol}{-} % Das Aufz�hlungszeichen f�r die folgende Liste
\cvlistdoubleitem{Open Source Software}{UNIX-Systeme}
\cvlistdoubleitem{Klavier}{Fotographie}
\cvlistdoubleitem{Literatur}{}

Berlin, \today\\ % Aktuelles Datum und Stadt
%\includegraphics[scale=0.7]{us.jpg}\\ % Unterschrift. L�schen falls nicht vorhanden
\end{document}


%% end of file `Lebenslauf_Template.tex'.
