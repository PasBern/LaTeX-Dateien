\documentclass[11pt,compress,handout,blackandwhite]{beamer}

\usepackage{beamerthemesplit}
\usepackage[ansinew]{inputenc}
\usepackage[ngerman]{babel}
\usepackage{xcolor}
\usepackage{fancybox}

\definecolor{links}{HTML}{2A1B81}
%%Farbe von Hyperlinks �ndern
\hypersetup{colorlinks,linkcolor=,urlcolor=links}
      



%%Pakete f�r mathematische Symbole und Umgebungen
\usepackage{amsmath,amsfonts,amssymb}

%%Paket f�r Bilder
\usepackage{epsfig}

\usepackage{marvosym}

\usepackage{textcomp}

\usepackage{hyperref}

\definecolor{ubuntu}{rgb}{0.99,0.5,0.2}
\definecolor{dark-violet}{rgb}{0.40,0,0.48}
\definecolor{dunkelgrau.60}{gray}{0.40}




\title{Paketverwaltung unter Ubuntu}
\author{Pascal Bernhard}
\institute[BeLUG]{Berliner Linux User Group}
\date{\today}

\titlegraphic{\includegraphics[width=3.8cm,height=1.2cm]{Black-Ubuntu-Logo.png}}
\logo{\pgfimage[width=1.9cm,height=0.6cm]{belug-ev.png}}

\begin{document}
%% Folie 1 (Titelfolie)
\frame{\titlepage}


\section{Softwareverwaltung unter Linux}


%--------------------------------------------------------------------------------------%
%%%Folie 2

%\subsection{1.Das Konzept Paketverwaltung}

\frame
{
  \frametitle{Was bedeutet Paketmanagement?}

  \begin{itemize}
  \item<1-> Linux-Software ist in Paketen organisiert
	\begin{itemize}
	 \item \textcolor{dunkelgrau.60}{Programme (z.B. Firefox) bestehen aus einer Mehrzahl an Paketen} 
	 \newline
	 \item \textcolor{dunkelgrau.60}{Pakete setzen sich aus mehreren Dateien zusammen}
	\end{itemize}

  \vspace{0.5cm}

  \item<2-> Pakete sind h�ufig voneinander abh�ngig
	\begin{itemize}
	 \item \textcolor{dunkelgrau.60}{Programme teilen sich Pakete}
	\end{itemize}
   
  \vspace{0.5cm}

  \item<3-> Softwareinstallation erfolgt �ber das Paketmanagement-Tool \textsl{\textcolor{dark-violet}{Apt}} (Ubuntu)
  \end{itemize}
}


%--------------------------------------------------------------------------------------%
%%%Folie 3


%\subsection{2.Ubuntu Software-Center}

\frame
{
  \frametitle{Wie installiere ich Programme?}

  \begin{itemize}
  \item<1-> Ubuntu Software-Center

  \vspace{1.5cm}

  \item<2-> Kommandozeilentool: \textsl{\textcolor{dark-violet}{Apt}}

  \vspace{1.5cm}
  
  \item<3-> GUI f�r Apt: \textsl{\textcolor{dark-violet}{Synaptic}}
  \end{itemize}
}


%--------------------------------------------------------------------------------------%
%%%Folie 4


%\subsection{3.Repositories}

\begin{frame}

  \frametitle{Was sind Repositories? Wo kommen die Pakete her?}
	
  \begin{itemize}
  \item<1-> Repositories sind Distributions-spezifisch
    
    \begin{itemize}
    \item \textcolor{dunkelgrau.60}{\Ovalbox{Repository bzw. Paketquelle} \textsl{Softwarearchiv mit Paketen speziell f�r eine Ubuntu-Version}}

    \item \textbf{\textcolor{ubuntu}{Bitte nur f�r Ubuntu gedachte Paketquellen verwenden!}}
    \end{itemize}


  \vspace{0.5cm}

  \item<2-> Konfiguration der Paketquellen in \texttt{\textcolor{dark-violet}{/etc/apt/sources.list}}

  \vspace{0.5cm}

  \item<3-> Repository-Bereichen unter Ubuntu

    \begin{itemize}
     \item \textcolor{dunkelgrau.60}{verschiedene Bereiche f�r unterschiedlich klassifizierte Pakete}
     \item \textcolor{blue}{main}: offiziell unterst�tzte Pakete mit freier Lizenz
     \item \textcolor{green}{restricted}: offiziell unterst�tzte Pakete, die nicht einer freien Lizenz unterliegen
     \item \textcolor{yellow}{universe}: von der Linux-Community unterst�tzte Pakete unter freier Lizenz
     \item \textcolor{magenta}{multiverse}: nicht-freie Software
    \end{itemize}


  \end{itemize}

\end{frame}
	
	
%--------------------------------------------------------------------------------------%
%%%Folie 5	
	
%\subsection{4.Updates}

\frame
{
  \frametitle{Wie halte ich mein Ubuntu auf dem aktuellen Stand?}
  
  \begin{itemize}
  \item<2-> Updates/Upgrades werden zentral �ber das Paketmanagement gemacht f�r alle Pakete

  \vspace{1.5cm}

  \item<3-> Update �ber Synaptic

  \vspace{1.5cm}

  \item<4-> Update auf der Kommandozeile mit \texttt{\textcolor{ubuntu}{sudo apt-get update}} \& \texttt{\textcolor{ubuntu}{sudo apt-get upgrade}}
\end{itemize}
}

%--------------------------------------------------------------------------------------%
%%%Folie 6

%\subsection{5.Installation/Entfernen von Paketen}

\begin{frame}[containsverbatim]
\frametitle{Wie installiere \& entferne ich Pakete unter Ubuntu?}

  \begin{itemize}
   \item Update der Paketquellen \texttt{\textcolor{ubuntu}{sudo apt-get update}}

   \vspace{1.5cm}

   \item Suche nach Paketen \texttt{\textcolor{ubuntu}{sudo apt-cache search PAKETNAME}}
   
   \vspace{1.5cm}

   \item Installation von Paketen \texttt{\textcolor{ubuntu}{sudo apt-get install PAKETNAME1 PAKETNAME2}}
  \end{itemize}

\end{frame}

%--------------------------------------------------------------------------------------%
%%%Folie 7

%\subsection{6.Sonderrepositories unter Ubuntu - PPAs}


\begin{frame}[containsverbatim]
\frametitle{PPAs: Personal Package Archives}

  \begin{itemize}
    \item nicht-offizielle Repositories - keine Unterst�tzung durch Ubuntu/Cannonical

    \vspace{1.5cm}

    \item Beispiel: \textcolor{dark-violet}{Ubuntu Tweak}

      \begin{itemize}
       \item \texttt{\textcolor{ubuntu}{sudo add-apt-repository ppa:tualatrix/ppa}}

       \item \texttt{\textcolor{ubuntu}{sudo add-get update}}

       \item \texttt{\textcolor{ubuntu}{sudo apt-get install ubuntu-tweak}}
      \end{itemize}



  \end{itemize}



\end{frame}


%--------------------------------------------------------------------------------------%
%%%Folie 8

%\subsection{7.Manuelle Installation von Paketen}
\begin{frame}

\frametitle{Begriffserkl�rung}
	  
      \begin{itemize}
      \item \textbf{Repository:} \textcolor{dunkelgrau.60}{Softwarearchiv mit Paketen speziell f�r eine Linux-Distribution}

      \vspace{1.0cm}

      \item \textbf{Paketabh�ngigkeiten:} \textcolor{dunkelgrau.60}{Programme setzen jeweils bestimmte Pakete voraus. Die Paketverwaltung hat die Aufgabe diese Abh�ngigkeiten zu managen. Werden die Paketabh�ngikeiten verletzt, weil zwei unterschiedliche Programme ein bestimmtes Paket in jeweils anderer Version ben�tigen, wird eines dieser Programme entweder entfernt, bzw. dieses erst gar nicht installiert.}
      \end{itemize}

\end{frame}


\begin{frame}

\frametitle{Begriffserkl�rung}
	  
      \begin{itemize}
	\item \textbf{propriet�re Software/Pakete:} \textcolor{dunkelgrau.60}{Propriet�re Software/Pakete stehen nicht unter einen freien Lizenz (z.B. GPL, LGPL, Afero-Lizenz, etc.) und k�nnen dementsprechend nicht wie freie Software im Quellcode eingesehen, ver�ndert und weitergegeben werden oder dies ist nur sehr eingeschr�nkt m�glich.}

	\vspace{0.5cm}

	\item Weitere Informationen:

	  \begin{itemize}
	  \item \textsl{Erl�uterung freier Software:} \url{https://www.gnu.org/philosophy/categories.html.en}{}
	  \item \textsl{Softwarelizenzen:} \\
	  \url{http://www.ifross.org/lizenz-center}{}
	  \item \textsl{Konzept der Vier Freiheiten:} \url{https://fsfe.org/about/basics/freesoftware.de.html}
	  \end{itemize}

      

      \end{itemize}



\end{frame}



\end{document}
