\documentclass[11pt,a4paper,titlepage]{moderncv}

% moderncv themes
\moderncvtheme[blue]{casual}                 % optional argument are 'blue' (default), 'orange', 'red', 'green', 'grey' and 'roman' (for roman fonts, instead of sans serif fonts)
%\moderncvtheme[green]{classic}                % idem

% character encoding
\usepackage[english,ngerman]{babel}
\usepackage[ansinew]{inputenc}                   % replace by the encoding you are using

\usepackage{marvosym}

% adjust the page margins
\usepackage[scale=0.8]{geometry}
\usepackage{graphicx}
%\setlength{\hintscolumnwidth}{3cm}						% if you want to change the width of the column with the dates
%\AtBeginDocument{\setlength{\maketitlenamewidth}{6cm}}  % only for the classic theme, if you want to change the width of your name placeholder (to leave more space for your address details
\AtBeginDocument{\recomputelengths}                     % required when changes are made to page layout lengths



%----------------------------------------------------------------------------------
%            Kontaktdaten
%----------------------------------------------------------------------------------
% VORNAME
\firstname{Pascal}
% NACHNAME
\familyname{Bernhard}
%TITEL (optional, ggf. einfach die Zeile l�schen!)
\title{Lebenslauf}    
%ADRESSE  (optional, ggf. einfach die Zeile l�schen!)
\address{Schwalbacher Strasse 7}{12161 Berlin}
%HANDYNUMMER  (optional, ggf. einfach die Zeile l�schen!)
\mobile{0152 / 38 50 23 63} 
%FESTNETZNUMMER  (optional, ggf. einfach die Zeile l�schen!)
\phone{030 / 32 66 58 00}
%EMAIL-ADRESSE  (optional, ggf. einfach die Zeile l�schen!)
\email{pascal.bernhard@belug.de} 
%HOMEPAGE  (optional, ggf. einfach die Zeile l�schen!)
\homepage{www.secunergy.com}
%FOTO  (optional, ggf. einfach die Zeile l�schen!)
%  64pt = H�he des Bildes, 'picture' = Name des Bildes
\photo[120pt]{sepia}

% to show numerical labels in the bibliography; only useful if you make citations in your resume
%\makeatletter
%\renewcommand*{\bibliographyitemlabel}{\@biblabel{\arabic{enumiv}}}
%\makeatother

%Spaltenbreite von \cvline �ndern:
\renewcommand*{\cvlanguage}[3]{%
\cvline{#1}{\begin{minipage}[t]{.225\maincolumnwidth}\textbf{#2}\end{minipage}%
\hfill%
\begin{minipage}[t]{.725\maincolumnwidth}\raggedleft\footnotesize\it shape #3\end{minipage}%
}
}

%Spaltenbreite von \cvline �ndern - zweiter Versuch
\setlength{\hintscolumnwidth}{3.5cm} 


%----------------------------------------------------------------------------------
%            Inhalt
%----------------------------------------------------------------------------------
\begin{document}


    \vspace*{\fill}
    
      \maketitle
      \section{Zu meiner Person}
	\cvline{Geboren}{am 08.04.1982 in Berlin}
	\cvline{Studienfach}{Politikwissenschaft}
	\cvline{Auslandserfahrung} {USA (1 Jahr) \& Frankreich (3 Jahre)}
	\cvline{Sprachkenntnisse}{Englisch \& Franz�sisch (verhandlungssicher)}
	\cvline{IT}{Linux, Office-Suite (MS Office, OpenOffice), Content Management Systeme (Contao, Joomla), LaTeX}

    \vspace*{\fill}



\newpage



\section{Praktische Erfahrungen}
\cvline{11/2011 - 3/2012}{Werkstudent \newline bei Startup-Unternehmen AKM3 in
Berlin (Search Engine Optimization) 				
\newline 
			\begin{itemize}
			  \item Suchmaschinenoptimierung von Webseiten
			\end{itemize} }

\cvline{6/2009 - 9/2011}{\textbf{Associate Consultant} \newline bei der
Unternehmensberatung SCI Verkehr GmbH in Berlin \newline
			\begin{itemize} 
			  \item Erstellung von Newslettern \& Unternehmensprofilen
			  \item Recherchearbeiten \& Erstellen von Marktstudien
			  \item Korrespondenz \& telef. Kontakt mit Kunden
			  \item B�roorganisation (Bestellung von Materialien, Arbeitsstundenplanung)
			\end{itemize} }

\cvline{11/2008 - 5/2009}{\textbf{Werkstudent} \newline bei Startup-Unternehmen
hiogi GmbH in Berlin (Anbieter von Dienstleistungen f�r mobile Ger�te) \newline
			\begin{itemize}
			  \item Sekretariatsarbeiten
			  \item Auswertung von Nutzer-Bewertungen
			  \item Informationsaustausch und Kontakt mit Nutzer-Community
			\end{itemize} }

\cvline{10/2008 - 4/2009}{\textbf{Praktikum} \newline am Forschungsinstitut
			  \textsl{Jean-Monnet Centre for European Integration}
			  in Berlin \newline
			\begin{itemize}
			  \item {Unterst�tzung des von der EU-Kommission in
				 Auftrag gegebenen Forschungsprojektes
				 \textsl{The Political Economy of EuroMed
				 Governance}: Datenrecherche, Verfassen von
				 L�nderberichten, Vorbereitung von Vortr�gen
				 und Konferenzen}
			  \item Sekretariatsarbeiten: Korrespondenz, Dokumentenverwaltung
			  \item Unterst�tzung der Forschungsaktivit�ten der
				hauptamtlichen Mitarbeiter: Erstellen von
				Pr�sentationen, Schaubildern sowie
				Excel-Tabellen 
			\end{itemize} }

\cvline{7/2006 - 7/2008}{\textbf{Praktikum/Charg\'{e} d'Affaires} \newline an
der American Chamber of Commerce in Paris \newline
			\begin{itemize}
			  \item Erstellung von monatlichem Newslettern
			  \item Mitarbeit an Marktstudien (Privatisierung von
				Infrastruktur \& Utilities in Frankreich)
			  \item Pflege der Kontaktdatenbank
			  \item Betreuung von Kunden
			\end{itemize} }

\cvline{4/2005 - 10/2005}{\textbf{Praktikum} \newline am Wissenschaftszentrum
f�r Sozialforschung Berlin \newline
			\begin{itemize}
			  \item Recherchearbeiten f�r Studie zu Arbeitsmarktreformen
			  \item Lehrt�tigkeit durch Tutorien f�r Studenten
			\end{itemize}
}

\bigskip 

\section{Studium}

\subsection{Hauptstudium "`Politikwissenschaft"'}
\cvline{ab 10/2012}{Diplomarbeit}
\cvline{Titel}{\emph{Energie-Versorgungssicherheit der Baltischen Staaten:
Chancen durch eine Europ�ische Energiepolitik}}
\cvline{Betreuer}{Dr. Prof. Schr�der \& Dr. Prof. Kerner}



\subsection{Hauptstudium "`Politikwissenschaft"'}
\cvline{seit 10/2009}{an der FU Berlin}



\subsection{Hauptstudium "`Master Finance et Strat�gie"'}
\cvline{9/2005 - 8/2007}{an Science Po Paris}



\subsection{Grundstudium "`Politikwissenschaft"'}
\cvline{9/2002 - 8/2005}{an der FU Berlin \emph{Note: 1,6}}
\cvline{13.06.2009}{Vordiplom}
\cvline{Titel}{\emph{Transformationsprozesse in Lettland}}
\cvline{Betreuer}{Dr. Prof. Kerner}

\bigskip

\section{Sprachkenntnisse}
\cvline{Deutsch}{Muttersprache}
\cvline{Englisch}{Verhandlungssicher in Wort und Schrift}
\cvline{Franz�sisch}{Verhandlungssicher in Wort und Schrift}
\cvline{Russisch}{Grundkenntnisse}

\bigskip

\section{Schulbildung}
\cvline{8/1992 - 7/2001}{Gymnasium Freiburg im Breisgau}
\cvline{8/1988 - 7/1992}{Grundschule Freiburg im Breisgau}

%\section{Erfahrung \& Kenntnisse}
%----------------------------------------------------------------------------------
%            ACHTUNG: Bei �nderungen das "}" am Ende nicht vergessen!!!
%----------------------------------------------------------------------------------
%\cvline{}{B�roorganisation 
%	  \newline Jean-Monnet Centre, American Chamber of Commerce \&
%	  \newline Unternehmensberatung SCI Verkehr}
%												 T�tigkeiten:
%												 % Aufz�hlung
%												 \begin{itemize}
%													\item %Entwicklung von Prototypen auf der Basis von PHP, (X)HTML, JavaScript (jQuery) und MySQL
%													\item %Entwicklung von Applikationen mittels PHP und XML
% 											 \end{itemize}
%}%-- Geh�rt zu \cvline
%\cvline{2/2011 - 7/2001}{Typo3 Developer \newline bei BummbBumm KG in Mannstedt \newline
%												 T�tigkeiten:
%												 % Aufz�hlung
%												 \begin{itemize}
%													\item %Optimierung der Templates und Struktur
%													\item %Beratung und Unterst�tzung der Redaktion, Vermarktung und der Marketingabteilung
%  											 \end{itemize}
%}%-- Geh�rt zu \cvline

%\subsection{Sonstiges}
%\cvline{11/2006 - 12/2008}{IT-Helpdesk Servicekraft \newline bei Musterverwaltung in Musterstadt}

\bigskip


\section{IT-Kenntnisse}
\cvline{}{GNU/Linux}
\cvline{}{MS Office Suite}
\cvline{}{LaTeX}
\cvline{}{Content Management Systeme (Joomla \& Contao)}

\bigskip

\section{Interessen}
\renewcommand{\listitemsymbol}{-} % Das Aufz�hlungszeichen f�r die folgende Liste
\cvlistdoubleitem{Open Source Software}{UNIX-Systeme}
\cvlistdoubleitem{Klavier}{Fotographie}
\cvlistdoubleitem{Literatur}{}

%Berlin, \today\\ % Aktuelles Datum und Stadt

\end{document}


%% end of file `Lebenslauf_Template.tex'.
