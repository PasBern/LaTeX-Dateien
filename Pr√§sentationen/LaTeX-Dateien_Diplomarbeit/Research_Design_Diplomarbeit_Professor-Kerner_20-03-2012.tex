\documentclass[11pt,a4paper,ngerman]{scrartcl}
\usepackage[T1]{fontenc}
\usepackage{parskip}
\pagestyle{empty}
\usepackage[ansinew]{inputenc}


\usepackage{xcolor}
\usepackage[ngerman]{babel}
\usepackage{relsize}
\usepackage{paralist}
\usepackage{typearea}
\usepackage{setspace}
\usepackage{textcomp}
\usepackage{layouts}
\usepackage{amsmath}
\usepackage{fancybox}
\usepackage{graphicx}
\usepackage{calc}



%%%vordefinierte Farben f�r Seitenfarben
%  \usepackage[usenames,dvipsnames]{color}

\definecolor{dunkelgrau.80}{gray}{0.20}
\definecolor{dunkelgrau.60}{gray}{0.40}
\definecolor{Dschungelgr\"{u}n}{cmyk}{0.99,0,0.52,0.2}
\definecolor{midblue}{rgb}{0.173,0.212,0.597}

%%%Seitenr�nder einstellen
\usepackage[top=10mm,bottom=10mm,left=15mm,right=35mm,marginparsep=12pt]{geometry}



%%%-----------------------------------------------------------------------
\begin{document}



 
 	\begin{flushright}
 			{\small \sffamily{\textcolor{dunkelgrau.60}{Fragestellung\\
 							\emph{Pascal Bernhard}\\
 							Diplomarbeit\\
 							\emph{Prof. Dr. Manfred Kerner}}}}
 	\end{flushright}						

	

	\begin{center}
			{\shadowbox{\sffamily{\textbf{ 
				Energie im Baltikum
			\texttwelveudash{} Mehr Versorgungssicherheit durch gemeinsame europ\"{a}ische Energiepolitik?}}}}
	\end{center} 	

	
	\textcolor{dunkelgrau.80}{
		
		\begin{itemize}
		
			
			
			\item{\sffamily{\textbf{\emph{Kann die EU-Energiepolitik die Versorgungssicherheit der baltischen L�nder verbessern?}}}}
 
			\normalmarginpar

 			\begin{itemize}
 			
 				\item{\textbf{Theorieanwendung:} Anwendung von IB-Theorien (liberaler Intergouvernementalismus vs. Neo-Realismus) \& Integrationsans\"{a}tzen auf europ�ische Energiepolitik am Beispiel der baltischen L�nder}


			      \begin{itemize}

			       \item{Mit welchem theoretischen Ansatz kann Energiepolitik im Bereich Versorgung mit Erdgas in Europa heute besser erkl�rt werden?}

				\begin{itemize}

				  \item{baltische L�nder haben eine zu schwache Position gegen�ber Russland und m�ssen somit auf eine Integration der Energiepolitik in der Europ�ischen Union setzen}

				  \item{der \textsl{Neo-Realismus} w�rde voraussagen, dass Energiepolitik in Europa weiterhin von den Nationalstaaten auf bilateraler Ebene geregelt werden\\
				  $\Rightarrow$ gro�e Mitglieder (Deutschland, Frankreich, Italien z.B.) schlie�en separat Liefervertr�ge mit Russland ab\\
				  $\Rightarrow$ schw�chere Mitglieder der EU sehen ihre Interessen missachtet}

				  \item{laut \textsl{Supranationalismus\textendash Institutionalismus} wird Energiepolitik in Europa nicht mehr von den Nationlstaaten sondern �ber gemeinschaftliche Institutionen bestimmt\\
				  $\Rightarrow$ auch die Anliegen der kleinen Mitglieder werden ber�cksichtigt}\\
				  $\Rightarrow$ ein umfassendes supranationales legislativ-exekutives Gef�ge ist hierf�r ausreichend
	    \



				\end{itemize}

			       \item{Welche Faktoren treiben die Integration von Energiepolitik auf europ�ischer Ebene?}

			      \end{itemize}



 				\item{\textsl{aktuelle politische Relevanz}}
 				
 					\begin{itemize}

 					\item{extreme Abh�ngigkeiten des Baltikums von russischem Gas} 					
 					\item{Energieversorgungssicherheit als gesamteurop\"{a}ische Herausforderung \texttwelveudash{} Energieversorgung in Osteuropa idealer  Testfall f�r Effektivit�t einer gemeinsamen europ�ischen Energiepolitik}

 					\item{Russland betreibt forsche bis aggressive Au�enpolitik gegen�ber seinen Nachbarn (Beispiel: {} \textsl{Cyberattacken gegen Estland  im Jahre 2007, Streitigkeiten �ber Gaslieferung nach Westeuropa �ber die Ukraine \& Gazproms Rolle auf dem litauischen Energiemarkt)}}
 					\end{itemize}


				\item{\textbf{Annahme: \emph{Aufgrund ihrer starken Abh�ngigkeit von russischen Gaslieferungen und schwachen Position in internationalen Verhandlungen, ist eine starke gemeinsame Energiepolitik von Interesse f�r die baltischen Staaten}}}
 					
 			\end{itemize}
 
		\end{itemize} 
 
 		\begin{itemize}
 			
 			\item{Theoretische Grundlagen}
 				
 				\begin{itemize}
 					
 					\item{Energie als klassischer Bereich f�r neo-realistische Ans�tze \texttwelveudash {} Energiesektor ist eng mit dem Nationalstaat verbunden}

 					\item{Erkl�rung der Integrationsfortschritte in der Europ�ischen Union auf dem Gebiet der Energiepolitik}
 						
 						\begin{itemize}
 							
 							\item{liberaler  Intergouvernementalismus} 
 							\item{Spill-over Effekte durch gemeinsamen Markt}
							\item{Problem kollektiver Handlungen \texttwelveudash {} \"{U}berwindung des Kooperationsdilemmas (Abwehr einer \textsl{divide-et-impera} \texttwelveudash {} Strategie der F\"{o}rderl\"{a}nder)}
 										
 						\end{itemize}
 	
 	
 	
 						
 
 	
 	
 							

 		
				  \end{itemize}
 			
 		\end{itemize}
 		
		}
 							
	\begin{itemize}
		
		
		\item{\textbf{\sffamily{\textsl{abh\"{a}ngige Variable:  Handlungsspielraum der EU-Energiepolitik}}}}	
			
			\begin{itemize}
				\item{Grad der Integration in Bezug auf Energiepolitik}
				\item{\textsl{Operationalisierung}}
							
					\begin{itemize}
						\item{\"{U}bertragung von Kompetenzen an europ\"{a}ische Organe samt Sanktionsinstrumente gegen widerwillige Mitgliedsstaaten}
						\item{Bewilligung von Mitteln zur Fortentwicklung eines einheitlichen europ�ischen Energiemarktes (Zusammenschluss von Netzen)}
						\item{politische Beschl\"{u}sse zur  Koordination nationaler Politiken, bzw. gemeinsame Verhandlungen mit Russland}
						\item{Umfang autonomer Handlungen einzelner Mitgliedsstaaten an der gemeinsamen Position vorbei}
					\end{itemize}
					
			\end{itemize}}
			
		\item{\textcolor{midblue}{\sffamily{\textbf{\textsl{unabh\"{a}ngige Variablen}}}}}
		
			\begin{itemize}
				\item{Pr\"{a}ferenzen  der Mitgliedsstaaten zu Integration von Energiepolitik}
				\item{Integrationswille der Gemeinschaftsorgane: {} Kommission (Vorschl�ge f�r neue Direktiven) \& Europ\"{a}ischen Parlament (Abstimmungsverhalten)}
				\item{\textsl{Operationalisierung:}}

									
				\begin{itemize}
					
					\item{\textcolor{dunkelgrau.60}{Kompetenzverteilung nach Europ�ischen Vertr�gen}}

					\item{\textcolor{dunkelgrau.60}{\textsl{Kommission:} {} Gr\"{u}nbuch und White Papers, Statements, Entw\"{u}rfe des DG TREN, Staff Working Documents}}			
																	\item{\textcolor{dunkelgrau.60}	
					{\textsl{Mitgliedsstaaten:} {} Stellungnahmen des Rates,  Protokolle (soweit zug\"{a}nglich), COREPER und Energiekommittee}}			
	
					\item{\textcolor{dunkelgrau.60}{\textsl{Energiepolitik im Baltikum:} {} Handeln die baltischen L�nder auch gemeinschaftlich? Gibt es auf regionaler Ebene Bestrebungen die Energiem�rkte miteinander zu koppeln und gemeinsame Infrastruktur zu Nutzen (LNG-Terminal f�r Gaslieferungen aus anderen L�ndern}}					
				\end{itemize}
							
			\end{itemize}
				
	\end{itemize} 						
	

\vspace{2cm}

%%		    \noindent\Ovalbox{%
%%		    \begin{minipage}{0.98\linewidth-2\fboxsep-2\fboxrule}%
%%		     \sffamily{% 
%%		    \begin{itemize}
%%
%%		       \item{\textsl{\textcolor{purple}
%%		      {K�nnen die baltischen L�nder als Eins gesehen werden?}}}
%%
%%		       \item{\textsl{\textcolor{purple}{Falls nicht, wie soll der Vergleich erfolgen?}}}
%%		       
%%		       \item{\textsl{\textcolor{purple}{Die Region als gemeinsamer Nenner? Welche gleichen 						Merkmale in Bezug auf Energie teilen die baltischen Staaten?}}}

%%		       \item{\textsl{\textcolor{purple}{Separate Behandlung der L�nder ist professionell, aber 						die Quellen sind schwer zug�nglich}}}
%%
%%		      \end{itemize}}

%%		    \end{minipage}}


							
\end{document}


%%%%%%%%%%%%%%%%%%%%%%%%%%%%%%%%%%%%%%%%%%%%%%%%%%%%%%%%%%%%%%%%%%%%%%%%%%%%%%%%%%%%%%%%


