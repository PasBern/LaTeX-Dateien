\documentclass[DIN, 10pt, pagenumber=false, parskip=half, fromalign=right, fromphone=true, fromemail=true, fromurl=true,              fromlogo=true, fromrule=false]{scrlttr2}

 
\usepackage[ansinew]{inputenc}
\usepackage{multicol}
\usepackage{rotating}
\usepackage{xcolor}
\usepackage[german,germanb]{babel}
\usepackage{fontenc}
\usepackage{textcomp}
\usepackage{marvosym}


%urspr�ngliche Pakete:
%\usepackage[latin1]{inputenc}
%\usepackage{german}

\RequirePackage{graphicx}

%Paket f�r farbige Tabellen
\usepackage{colortbl}

%%%Serifenlose Schrift f�r das gesamte Dokument
\renewcommand{\familydefault}{\sfdefault}


 

\setkomavar{fromname}{DAK}

\setkomavar{fromaddress}{22778 Hamburg}

\setkomavar{fromphone}{030/911292980}

\setkomavar{fromfax}{030/9120298-7090}

\setkomavar{fromemail}{service723300@dak.de}

%\setkomavar{fromurl}{www.dak.de}

\setkomavar{signature}{Hans Meyer}

\setkomavar{fromlogo}{\includegraphics*[width=2cm]{DAK-Gesundheit_logo.png}}

 

\begin{document}

 

\begin{letter}{Herr \\ Pascal Bernhard \\
               Schwalbacher Stra�e 7 \\ 
	       12161 Berlin}

Sehr geehrter Herr Bernhard,

Sie hatten im Zeitraum vom 11. April 2011 bis einschlie�lich 30. November 2012 gem�� V.
Sozialgesetzbuch \S 44 Abs. 1  und \S 46 Abs. 2 von der DAK Krankengeld bezogen, da Sie aufgrund
Ihrer Tumorerkrankung laut �rztlichem Attest vom 14. April 2011 als arbeitsunf�hig eingestuft
wurden. Wir mussten nun feststellen, dass Sie zum Wintersemester 2011/2012 am 1. Oktober 2011 Ihr
unterbrochenes Studium an der Freien Universit�t Berlin wieder aufgenommen haben und auch im
darauffolgende Sommersemester 2012 als regul�rer Student eingeschrieben waren. Eine Krankschreibung
schlie�t jedoch ein Vollzeitstudium zum gleichen Zeitpunkt aus (siehe II. Sozialgesetzbuch \S 7 und
\S 8 Abs. 1). Da Sie ab dem 1. Oktober 2011 offenbar wieder studierf�hig waren, war ab diesem Datum
die bescheinigte Arbeitsunf�higkeit nicht mehr gegeben. Entsprechend bestand im Zeitraum vom 1.
Oktober 2011 bis 30. November 2012 kein Anspruch auf Krankengeld. Wir fordern Sie hiermit auf, die
unberechtigt erhaltenen Leistungen innerhalb von 4 Wochen 
zur�ckzuerstatten. Den genauen Betrag entnehmen Sie bitte der folgenden Tabelle:

\vspace{8mm}

%Tabelle
\setlength{\tabcolsep}{23pt}
\renewcommand{\arraystretch}{1.4}
\definecolor{dunkelgrau}{rgb}{0.82,0.82,0.82}
\definecolor{hellgrau}{rgb}{0.92,0.92,0.92}


  \begin{tabular}[c]{>{\columncolor{dunkelgrau}}l >{\columncolor{hellgrau}}c r}
    \rowcolor{dunkelgrau}

    \multicolumn{3}{l}{\textbf{Leistungen: Krankengeld} \footnotesize{\textsl{(nach V.
Sozialgesetzbuch �44
Abs. 1 und �46 Abs. 2)}}} \\
    
    \rowcolor{dunkelgrau}
%   & & \\
    Bezugszeitraum: & monatlich & Summe \\

    \textsl{01.10.2011 -- 31.12.2011} & 935,96 \EUR & 2807,88 \EUR{} \\
    \textsl{01.01.2012 -- 31.10.2012} & 962,25 \EUR & 9622,50 \EUR{} \\
    \hline
    \cellcolor{white} \textbf{Gesamtbetrag:} & \cellcolor{white} & 12430,38 \EUR{} \\
    
  \end{tabular}

Mit freundlichem Gruss

\end{letter}

% Define a new letter foot
% (with tricky use of tabulars but without centering middle tabular;
% if you need to center the middle tabular, use \rlap to set first 
% tabular and \llap to set third tabular):
%\firstfoot{\scriptsize
%  \rlap{%
%  \begin{tabular}{@{}l@{}}
%    Busverbindungen\\
%    \usekomavar{bus}\\
%  \end{tabular}%
%  }\null
%  \hfill
%  \begin{tabular}{@{}l@{}}
%    Paketanschrift\\
%    \usekomavar{fromaddress}\\
%  \end{tabular}%
%  \hfill
%  \llap{%
%  \begin{tabular}{@{}l@{}}
%    \usekomavar*{frombank}\\
%    \usekomavar{frombank}\\
%  \end{tabular}
%  }\null
%}
 

\end{document}